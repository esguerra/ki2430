\documentclass[10pt, oneside, pdftex]{article}
\renewcommand{\thefootnote}{\roman{footnote}}
\usepackage{amsmath, amstext, amsfonts, amssymb}
\usepackage[pdftex]{graphicx}
\usepackage{subfigure, longtable}
\usepackage{float}
\usepackage{rotating}
%\usepackage{slashbox}
\usepackage{longtable}
\usepackage{url}
%\usepackage[left=2.54cm,bottom=2.54cm,top=2.54cm,right=2.54cm]{geometry}
\usepackage{fancyvrb}
\usepackage{setspace}
\usepackage{pdflscape}
\usepackage{threeparttable}
\usepackage{listings}
\usepackage{color}
\definecolor{dkgreen}{rgb}{0,0.6,0}
\definecolor{gray}{rgb}{0.5,0.5,0.5}
\definecolor{mauve}{rgb}{0.58,0,0.82}

\lstset{frame=tb, language=bash, aboveskip=3mm,
belowskip=3mm, showstringspaces=false, columns=flexible,
basicstyle={\small\ttfamily}, numbers=none,
numberstyle=\tiny\color{gray}, keywordstyle=\color{blue},
commentstyle=\color{dkgreen},stringstyle=\color{mauve},
breaklines=true,breakatwhitespace=true,tabsize=3}
%\usepackage[style=reading, sorting=ynt]{biblatex}
\singlespacing
\usepackage[left=2.54cm,bottom=2.00cm,top=2.00cm,right=2.54cm]{geometry}
%\usepackage[left=1.5in,bottom=1.0in,top=1.0in,right=1.0in]{geometry}
%\usepackage{chapterbib}
%\includeonly{chapter1}
%\usepackage{pdflscape}

%%%%%%%%%%%%%%%%%%%%%%%%%%%%%%%%%%%%%%%%%%%%%%%%%%%%%%%%%%%%%%%%%%%%%%%%%%%%%%%%
%% Fonts in Document
%%%%%%%%%%%%%%%%%%%%%%%%%%%%%%%%%%%%%%%%%%%%%%%%%%%%%%%%%%%%%%%%%%%%%%%%%%%%%%%%
%\usepackage{times}
%\usepackage{pslatex}
%\usepackage{newcent}
%\usepackage{palatcm}
%\usepackage{palatino}
\usepackage[T1]{fontenc}
\usepackage[scaled]{helvet}
\renewcommand*\familydefault{\sfdefault}
%\renewcommand*{\encodingdefault}{T1}
% For more options go to: http://www.tug.dk/FontCatalogue/


%%%%%%%%%%%%%%%%%%%%%%%%%%%%%%%%%%%%%%%%%%%%%%%%%%%%%%%%%%%%%%%%%%%%%%%%%%%%%%%%
% Begin Document
%%%%%%%%%%%%%%%%%%%%%%%%%%%%%%%%%%%%%%%%%%%%%%%%%%%%%%%%%%%%%%%%%%%%%%%%%%%%%%%%

\title{Using 3DNA: A Mini-Tutorial}
\author{Mauricio Esguerra\\
%\small  Department of Biosciences and Nutrition\\[-0.8ex]
%\small  Karolinska Institutet\\[-0.8ex]
%\small Huddinge SE-14157, Sweden\\[-0.8ex]
%\small  \texttt{mauricio.esguerra@ki.se}}
\small  Department of Cell and Molecular Biology\\[-0.8ex]
\small  Uppsala University\\[-0.8ex]
\small  Husargatan 3, Box 596, 751 24, Uppsala, Sweden\\[-0.8ex]
\small  \texttt{mauricio.esguerra@gmail.com}}  
\date{\today}
%\date{April 20, 2011}

\begin{document}
\maketitle
\tableofcontents
\bibliographystyle{nar}


\section{Installation}
To install 3DNA you first need to register in the x3dna forum and then
login as a registered user and head to:\\
\url{http://forum.x3dna.org/downloads/3dna-download/}

\noindent Where you will be able to download compiled versions of 3DNA
for the Linux, Mac-OS, and Windows operating systems. The compiled
versions also contain the source code in ansi-C.
If  you download  the  Linux  64-bit version  yourfile  will be  named
something like:
\textbf{x3dna-v2.3-linux-64bit.tar.gz}\\

You will need to uncompress this file by issuing the following command
in the terminal:
\begin{Verbatim}
tar -xvzf x3dna-v2.3-linux-64bit.tar.gz
\end{Verbatim}
This will create a new folder named \textbf{x3dna-v2.3} which you will
have to include  in your environment variables by  changing your shell
environment  configuration   file  named  ``.bashrc''  in   bash,  and
``.chsrc'' in csh  and tcsh.\\ If you are using  .bashrc the following
lines will need to be added:
\begin{Verbatim}
export X3DNA=/home/yourusername/x3dna-v2.3
export PATH=$PATH:$X3DNA/bin
\end{Verbatim}
A detailed explanation  of how to configure X3DNA can  be found in the
pdf   file  \textbf{x3dna\_v1.5.pdf}   found  in   the  x3dna-v2.3/doc
folder. If you're stuck in the  installation process feel free to drop
me a line at \url{mauricio.esguerra@gmail.com}

\section{fiber}
The \textrm{fiber} program included in  the 3DNA package allows you to
easily  generate  protein  data  bank  (pdb)  files  of  fiber  models
corresponding to common nucleic acid conformations, for example A-DNA,
B-DNA, and  Z-DNA.  To get a  list of all possible  fiber models which
you can reconstruct using \textrm{fiber}, type:
\begin{Verbatim}
fiber -l
\end{Verbatim}
If you type \textrm{fiber} alone in the terminal you will get a usage
message.\\
Stop and read the usage message carefully.\\

\noindent Now, you know that the crystal structures of A-RNA and B-DNA
must have a total of 11  and 10 residues per turn respectively, so, go
ahead and use  \textrm{fiber} to make an A-RNA with  12 residues and a
B-DNA with 11 residues.   Using a molecular visualization program such
as pymol, vmd, or chimera check the generated structure to corroborate
that you have the correct number of residues per turn.

\section{find\_pair}
The \textrm{find\_pair} program of 3DNA  allows you to find pairs in a
given pdb file.
To read the usage message open a terminal and type:
\begin{Verbatim}
find_pair
\end{Verbatim} 
Now go  to the protein data bank website  at \url{http://www.pdb.org}
and download the  structure with PDB\_ID: 1ehz. This  is the structure
for yeast phenylalanine tRNA. 
Run it through \textrm{find\_pair}:
\begin{Verbatim}
find_pair 1ehz.pdb 1ehz.inp
\end{Verbatim}
The ouput  will be a file  named \textit{1ehz.inp}. Go  ahead and open
the  resulting  file in  a  text  editor and  you  will  see that  the
\textrm{find\_pair}  program  has  found  the  canonical  Watson-Crick
base-pairs   G$\cdot$C  and  A$\cdot$U,   as  well   as  non-canonical
base-pairs such as G$\cdot$U and A$\cdot$G.

\section{analyze}
The \textrm{analyze} program  of the 3DNA software  package allows you
to  compute the  base-pair, and  base-pair step  parameters associated
with  a  given nucleic  acid  structure.  Analyze provides  additional
analysis features such as computing  the overlap areas between stacked
nucleic acid bases, sugar  conformations, and sugar-phosphate backbone
torsion angles.\\
A requirement of the \textrm{analyze} program  is that you have to run
the \textrm{find\_pair} program beforehand.\\
Go ahead and read the usage message of the analyze program by typing:
\begin{Verbatim}
analyze --help
\end{Verbatim}
in your terminal.\\

\noindent Now run the analyze  program on the previously produced
file using the \textrm{find\_pair} program:
\begin{Verbatim}
analyze 1ehz.inp
\end{Verbatim}
Notice that  you can run the  \textrm{find\_pair} and \textrm{analyze}
programs in one line by issuing the command:
\begin{Verbatim}
find_pair 1ehz.pdb stdout | analyze
\end{Verbatim}
This command  will produce a  wealth of files with  derived structural
information, but  the main one  will just have  the name of  the input
file followed by  the .out extension.  That is, you  should now have a
\textit{1ehz.out} file in your folder. Go ahead and take a look at the
\textit{1ehz.out} file in your favorite text editor.

\section{rebuild}
The very useful 3DNA program called \textrm{rebuild} can do just that,
rebuild nucleic acid structures. It uses as input either a set of base
step, or base-pair step parameters,  or a set of helical parameters to
create  a   pdb  file.   To   test  the  rebuild  program   first  run
\textrm{find\_pair}  and \textrm{analyze} on  the A-RNA  structure you
previously created:
\begin{Verbatim}
find_pair A-RNA.pdb stdout | analyze
\end{Verbatim}
Issuing  the  previous  command  must   have  created  a  file  called
\textit{bp\_step.par}.
Open this file in a text editor  and change the Twist values from 31.5
degrees to  a smaller value, say  25.0 degrees, and save  the new file
with the name \textit{undertwisted.par}.\\
Rebuild the undertwisted  RNA by typing the following  command in your
good-ole terminal:
\begin{Verbatim}
rebuild -atomic undertwisted.par undertwisted.pdb
\end{Verbatim}
This   will  generate   a  pdb   structure  with   no  sugar-phosphate
backbone. To add a sugar-phosphate backbone you will need to issue the
\textrm{cp\_std}  command  before   you  perform  the  \textrm{rebuild}
command, in the following way:
\begin{Verbatim}
cp_std RNA
rebuild -atomic undertwisted.par undertwisted.pdb
\end{Verbatim} 
Now rebuild a  pdb file using the unmmodified base-pair
step  parameters, that  is, using  the  original \textit{bp\_step.par}
file. Open  both structures in  a molecular visualization  program and
confirm that one helix is undertwisted with respect to the original.

\section{blocview}
Blocview is a perl script  which calls various 3DNA functionalities as
well  as  raster3D, molscript,  and  pymol  to  generate an  automated
picture. Since  blocview is just a  script it can  be easily modified,
for example, following row 83 which invokes get\_part -c (an
undocumented command that creates a molscript header) one  can add a
line to change the inner color of protein ribbons:
\begin{Verbatim}
# change inner color of ribbons from default 0.8
system("sed 's/grey 0.8/grey 0.5/g' temp2a > temp2");
\end{Verbatim}

\section{Exercises}
\begin{enumerate}
\item{Run   \textrm{find\_pair}  and  \textrm{analyze}   on  your  A-RNA
structure and modify the  resulting \textit{bp\_step.par} file so that
the sequence is AAAGGGUUUCCC instead of AAAAAAAAAAAA.\\
Use rebuild to generate a pdb file with the AAAGGGUUUCCC sequence.}
\item{Download the file with PBD\_ID: 1aoi, that is, chromatin's
nucleosome core particle,  from the protein data bank  website. Run it
through \textrm{find\_pair} and \textrm{analyze}.}
\item{Produce a  ``publication quality''  plot of Slide  vs. base-pair
    step number  for the chromatin  parameters you've computed  in the
    previous exercise.  Ask around (e.g.  drop me a line) for which solutions
    are  good  for  such  a  task.\\ I  recommend  \textrm{xmgrace},  the
   \textrm{matplotlib} library of \textrm{python}, and  \textrm{igor-pro}.}
\item{Repeat the recipe for Twist vs. base-pair step number (bpsn), and Roll
    vs. bpsn.}
\item{Compare your  graphs to  Figure 2 of  the journal  article ``DNA
    Sequence-Directed   Organization  of   Chromatin:  Structure-Based
    Computational    Analysis   of    Nucleosome-Binding   Sequences''
    \textit{Biophysical Journal}, \textbf{2009}, 96, 2245-2260.\\  What
    regularities do  you see?\\  How would you  describe the  DNA wrapping
    around the nucleosome in terms of Sliding, Twisting, and Rolling?}
\item{Go  to the  website: \url{http://w3dna.rutgers.edu}  and analyze
    the structure of your choice, for example the large subunit of the
  ribosome.}
\end{enumerate}

\section{Config}
In the config folder one can  find most of the files for configuration
of 3DNA internals. The following are descriptions of what they do.

\subsection{col\_chain.dat}
This file specifies color defaults  for the molscript files which 3DNA
produces. I don't  like to have purple proteins, so  the first thing I
do i this  file is to change the default protein  color from purple to
blue. I also dislike having a  wide tube to represent the nucleic acid
backbone so I change the value from 0.8 (default) to 0.5

\subsection{my\_header\_hres.r3d}
The header file for the raster3d objects. 

\subsection{my\_raster3d.par}
The name is self-explanatory.


\section{Interaction with gromacs}
Gromacs will read  pdb files in PDB  v3 format, so, for  a pdb created
with 3DNA to work with gromacs one has to do:

\begin{Verbatim}
pdb_get 355d
get_part -pdbv3 355d.pdb 355d-only-dna.pdb
pdb2gmx -f 355d-only-dna.pdb -o dna.gro -ff amber03 -water tip3p -p dna.top -i posre.itp
editconf -f dna.gro -o dna.gro -d 0.85
genbox -cp dna.gro -cs -o dna_b4em.gro -p dna.top

Here a miracle occurs KABUM! to create the steep.mdp that is!

grompp -f steep.mdp -c dna_b4em.gro -p dna -o dna.em
mdrun -nice 4 -s dna.em.tpr -o dna.em.tpr -c dna_b4em.gro -v
trjconv -f dna.em.tpr.trr -s dna.em.tpr -o dnatraj.pdb 
\end{Verbatim}

\bibliography{biblio}
\end{document}
