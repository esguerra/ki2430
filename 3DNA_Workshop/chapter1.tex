\maketitle
\bibliographystyle{nar}
\label{basics}
In the following guide we'll show  a step by step introduction to 3DNA
from installation of the software to practical examples.

\section{Installation}
To install 3DNA you will first have to download the software from:\\

\url{http://adna.rutgers.edu}\\

\noindent The webpage will ask you to register before allowing you to
download the program.\\
Once   registered    you'll   be    able   to   download    the   file
\textbf{Linux\_X3DNA\_v2.0.tar.gz} if you  selected the Linux version.\\
To uncompress, issue the following command in a terminal:
\begin{Verbatim}
tar -xvzf Linux_X3DNA_v2.0.tar.gz
\end{Verbatim}
This will create a new folder named \textbf{X3DNA} which you will have
to  include  in your  environment  variables  by  changing your  shell
environment  configuration   file  named  ``.bashrc''   in  bash,  and
``.chsrc'' in csh and tcsh.\\ 
If you are using .bashrc the following lines will need to be added:
\begin{Verbatim}
export X3DNA=/home/yourusername/X3DNA
export PATH=$PATH:$X3DNA/bin
\end{Verbatim}
A detailed explanation  of how to configure X3DNA can  be found in the
pdf file  \textbf{x3dna\_v1.5.pdf} found  in the X3DNA/doc  folder. If
you're stuck in the installation process feel free to drop me a line at
\url{mauricio.esguerra@gmail.com}

\section{fiber}
The \textrm{fiber} program included in  the 3DNA package allows you to
easily  generate  protein  data  bank  (pdb)  files  of  fiber  models
corresponding to common nucleic acid conformations, for example A-DNA,
B-DNA, and  Z-DNA.  To get a  list of all possible  fiber models which
you can reconstruct using \textrm{fiber}, type:
\begin{Verbatim}
fiber -l
\end{Verbatim}
If you type \textrm{fiber} alone in the terminal you will get a usage
message.\\
Stop and read the usage message carefully.\\

\noindent Now, you know that the crystal structures of A-RNA and B-DNA
must have a total of 11  and 10 residues per turn respectively, so, go
ahead and use  \textrm{fiber} to make an A-RNA with  12 residues and a
B-DNA with 11 residues.   Using a molecular visualization program such
as pymol, vmd, or chimera check the generated structure to corroborate
that you have the correct number of residues per turn.

\section{find\_pair}
The \textrm{find\_pair} program of 3DNA  allows you to find pairs in a
given pdb file.
To read the usage message open a terminal and type:
\begin{Verbatim}
find_pair
\end{Verbatim} 
Now go  to the protein data bank website  at \url{http://www.pdb.org}
and download the  structure with PDB\_ID: 1ehz. This  is the structure
for yeast phenylalanine tRNA. 
Run it through \textrm{find\_pair}:
\begin{Verbatim}
find_pair 1ehz.pdb 1ehz.inp
\end{Verbatim}
The ouput  will be a file  named \textit{1ehz.inp}. Go  ahead and open
the  resulting  file in  a  text  editor and  you  will  see that  the
\textrm{find\_pair}  program  has  found  the  canonical  Watson-Crick
base-pairs   G$\cdot$C  and  A$\cdot$U,   as  well   as  non-canonical
base-pairs such as G$\cdot$U and A$\cdot$G.

\section{analyze}
The \textrm{analyze}  program of the 3DNA software  package allows you
to  compute the  base-pair, and  base-pair step  parameters associated
with a  given nucleic acid structure. Analyze  provides additional analysis
features such as computing the overlap areas between stacked nucleic acid
bases,  sugar  conformations,  and  sugar-phosphate  backbone  torsion
angles.\\ 
A  requirement of the  \textrm{analyze} program is  that you
have to run the \textrm{find\_pair} program beforehand.\\ 
Go ahead and read the usage message of the analyze program by typing:
\begin{Verbatim}
analyze --help
\end{Verbatim}
in your terminal.\\

\noindent Now run the analyze  program on the previously produced
file using the \textrm{find\_pair} program:
\begin{Verbatim}
analyze 1ehz.inp
\end{Verbatim}
Notice that  you can run the  \textrm{find\_pair} and \textrm{analyze}
programs in one line by issuing the command:
\begin{Verbatim}
find_pair 1ehz.pdb stdout | analyze
\end{Verbatim}
This command  will produce a  wealth of files with  derived structural
information, but  the main one  will just have  the name of  the input
file followed by  the .out extension.  That is, you  should now have a
\textit{1ehz.out} file in your folder. Go ahead and take a look at the
\textit{1ehz.out} file in your favorite text editor.

\section{rebuild}
The very useful 3DNA program called \textrm{rebuild} can do just that,
rebuild nucleic acid structures. It uses as input either a set of base
step, or base-pair step parameters,  or a set of helical parameters to
create  a   pdb  file.   To   test  the  rebuild  program   first  run
\textrm{find\_pair}  and \textrm{analyze} on  the A-RNA  structure you
previously created:
\begin{Verbatim}
find_pair A-RNA.pdb stdout | analyze
\end{Verbatim}
Issuing  the previous command must  have created  a file called
\textit{bp\_step.par}.
Open this file in a text editor and change the Twist values from 31.5 degrees
to a smaller value, say 25.0 degrees, and save the new file with the name
\textit{undertwisted.par}.\\
Rebuild the undertwisted RNA by typing the following command in your good-ole
terminal:
\begin{Verbatim}
rebuild -atomic undertwisted.par undertwisted.pdb
\end{Verbatim}
This   will  generate   a  pdb   structure  with   no  sugar-phosphate
backbone. To add a sugar-phosphate backbone you will need to issue the
\textrm{cp\_std}  command  before   you  perform  the  \textrm{rebuild}
command, in the following way:
\begin{Verbatim}
cp_std RNA
rebuild -atomic undertwisted.par undertwisted.pdb
\end{Verbatim} 
Now rebuild a  pdb file using the unmmodified base-pair
step  parameters, that  is, using  the  original \textit{bp\_step.par}
file. Open  both structures in  a molecular visualization  program and
confirm that one helix is undertwisted with respect to the original.

\section{Exercises}
\begin{enumerate}
\item{Run   \textrm{find\_pair}  and  \textrm{analyze}   on  your  A-RNA
structure and modify the  resulting \textit{bp\_step.par} file so that
the sequence is AAAGGGUUUCCC instead of AAAAAAAAAAAA.\\
Use rebuild to generate a pdb file with the AAAGGGUUUCCC sequence.}
\item{Download the file with PBD\_ID: 1aoi, that is, chromatin's
nucleosome core particle,  from the protein data bank  website. Run it
through \textrm{find\_pair} and \textrm{analyze}.}
\item{Produce a  ``publication quality''  plot of Slide  vs. base-pair
    step number  for the chromatin  parameters you've computed  in the
    previous exercise.  Ask around (e.g.  drop me a line) for which solutions
    are  good  for  such  a  task.\\ I  recommend  \textrm{xmgrace},  the
   \textrm{matplotlib} library of \textrm{python}, and  \textrm{igor-pro}.}
\item{Repeat the recipe for Twist vs. base-pair step number (bpsn), and Roll
    vs. bpsn.}
\item{Compare your  graphs to  Figure 2 of  the journal  article ``DNA
    Sequence-Directed   Organization  of   Chromatin:  Structure-Based
    Computational    Analysis   of    Nucleosome-Binding   Sequences''
    \textit{Biophysical Journal}, \textbf{2009}, 96, 2245-2260.\\  What
    regularities do  you see?\\  How would you  describe the  DNA wrapping
    around the nucleosome in terms of Sliding, Twisting, and Rolling?}
\item{Go  to the  website: \url{http://w3dna.rutgers.edu}  and analyze
    the structure of your choice, for example the large subunit of the
  ribosome.}
\end{enumerate}



\bibliography{biblio}




